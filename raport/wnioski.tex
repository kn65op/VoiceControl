\chapter{Wyniki i wnioski}
\label{cha:wnioski}

Zagadnienie rozpoznawania pojedynczych fonemów jest trudne. Przedstawiony algorytm może być jedynie początkiem do dalszych prac, które mogłby usprawnić jego działania. Posiada on wiele wad oraz efekty jego działania odbiegają od ideału.

Skuteczność rozpoznawania wzorcowych słów określonych w postoci fonetycznej jest mała, ale nie jest zerowa. Podczas eksperymentów wyraz ,,karta'' zostawał rozpoznany w 50\% przypadków. 

Złożoność obliczeniowa jest bardzo duża, co powoduje, że już przy kilku wyrazach czas oczekiwania przekracza kilka sekund.

W dalszych pracach polecane jest rozpoczęcie od wyszukania parametrów, które byłyby w stanie poprawniej i dokładniej klasyfikować fonemy. Poprawić należy również sam algorytm dopasowania wzorców, ale ponieważ jest on ostatnim elementem przetwarzania powinno być opracowane na końcu w przypadku poprawiania działania całego algorytmu.
