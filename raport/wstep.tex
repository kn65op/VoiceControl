\chapter{Wstęp}
\label{cha:wstep}

Celem pracy było zrealizowanie, w~pełni konfigurowalnego sterowania głosowego w~środowisku GNOME~3. Pod pojęciem w~pełni konfigurowalnego przyjmuje się możliwość dowolnego zdefiniowania rozpoznawanych wyrazów oraz przyporządkowanych akcji. Ideałem była by możliwość wpisywania rozpoznawanych słów w~sposób ?gramatyczny?. W~zrealizowanym programie nie udało się tego osiągnąć i~słowa wpisywane są fonetycznie.

Metoda rozpoznawania poszczególnych fonemów miała być zrealizowana w~sposób opisany w ?odwołanie?. Niestety ta metoda nie dała spodziewanych rezultatów i~rozpoznawanie zostało zrealizowane za pomocą autorskiego algorytmu opisanego w~tej pracy.

Zawartość tej pracy to: rodział \ref{cha:nieudane} przedstawia realizację algorytmu opisanego w ?odwołanie?, rodział \ref{cha:algorytm} prezentuje wykorzystany algorytm, rodział \ref{cha:wnioski} przedstawia wyniki oraz wnioski.
