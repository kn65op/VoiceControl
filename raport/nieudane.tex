\chapter{Realizacja algorytmu z~artykułu}
\label{cha:nieudane}

Algorytm jest podzielony na kilka części, które są przeprowadzane szeregowo. Są to: podział mowy na fonemy, wyliczenie cech z danego fragmentu oraz rozpoznanie. Wszystkie są opisane ponieżej. Następną częścią powinno być dopasowanie znalezionych fonemów do zdefiniowanych wzorców, ale nie została zaimplementowana, ponieważ nie udało się uzyskać odpowiednich wyników w częściach wcześniejszych.

\section{Rozdzielenie fonemów}
\label{sec:rozdzielenie}

Przy rozpoznawaniu pojedynczych fonemów pierwszą czynnością, którą należy wykonać jest podział sygnału mowy na pojedyncze fonemy. W tym celu wyliczane jest ALCR ?odowłanie?. ALCR jest to ,,Average Level Crossing Rate''. Jest to średnia liczba przecięcia ustalonych progów przez sygnał mowy. Wykorzystany sposób wyliczenia ALCR jest następujący:

\begin{itemize}
    \item Wyznacznie progów. Do tego celu użyto rekomendacji w~?odowłanie? dotyczącej rozmieszczenia progów. Największa gęstość progów powinna być w~pobliżu zera oraz w~pobliżu maksymalnej wartości. Natomiast im bliżej środka, tym bardziej gęstość progów maleje.
    \item Wyliczenie LCR (,,Level Crossing Rate''). Dla każdej próbki wyliczana jest liczba progów, które zostały przecięte przez linię łączącą daną próbkę z~próbką poprzednią. W~oryginalnym algorytmie ta wartość była uśredniana w~pewnym otoczeniu, dla każdego progu. W implementacji ten krok został pomienięty, ponieważ znacznie zwiększał czas trwania obliczeń, a~poprawa działania algorytmu znikoma, jeśli w~ogóle istniałą.
    \item Wyliczenie ALCR. ACLR jest to średnia z~wyliczonych w~poprzednim kroku wartości z~pewnego zakresu. W~opisie algorymtu oraz w implementacji użyto otoczenia o~promieniu 100 próbek.
\end{itemize}

Po wyliczeniu ALCR należy znaleźć granice fonemów. Tymi granicami są minima lokalne funkcji ALCR, które spełniają następujące warunki:
\begin{itemize}
    \item Minimalna długość trwania fonemu to 12ms.
    \item Minimum lokalne musi być minimum w zakresie 20ms.
\end{itemize}

Znalezione w~ten sposób minima są punktami wyznaczającymi granice fonemów.

\section{Wyliczenie cech}
\label{sec:cechybin}

Ten etap jest przeprowadzany tylko dla tych fragmentów sygnału wyznaczonych w~poprzednim etapie, w których stwierdzono obecność mowy. Określanie obecności mowy odbywa się za pomocą obliczania energii fragmentu sygnału.

Wektorem cech jest wektor binarny otrzymywany z PSD (,,Power Spectral Density'').

PSD jest obliczane jako szybka transformata Fouriera funkcji autokorelacji sygnału wejściowego. Ponieważ ustalona wielkość wektora to 513 bitów, to autokorelacji podlega jedynie 512 początkowych próbek w danym fonemie (które powinny być reprezentatywne). Po operacji autokorelacji do transformacji Fouriera używanych jest 1024 próbki. Po zastosowaniu transformacji Fouriera wybieranych jest 513 próbek, poddanych wcześniej operacji obliczania modułu.

Wyliczone w~ten sposób PSD jest poddawne operacji progowania. Niestety nie przedstawiono w artykule sposobu dobierania wartość progu. Prób został przyjęty jako 20\% wartości maksymalnej PSD. Wartościom powyżej progu przypisywano 1, a~ponieżej progu 0. W ten sposób otrzymano wektor cech.

Na tym etapie zauważono, że kształ PSD dla fonemu ,,s'' odbiega od przedstawionego w ?odwołanie?.

\section{Rozpoznanie}
\label{sec:rozpoznaniebin}

Do rozpoznawanio użyto sieci neuronowej typu ,,Back Propagation'' o 513 wejściach oraz 34 wyjściach (po jednym dla każdego fonemu). Próby uzyskania optymalnej struktury sieci nie udały się. Probowano sieci dwu- oraz trzywarstwiwe o różnych liczbach neuronów ukrytych, ale żadna sieć osiągnęła wystarczająco niskiego błędu. Sieci o najmniejszych błędach dla danych uczących nie były w stanie rozpoznawać poprawnie fonemów, nawet z ograniczonego zbioru.

W~związku z~brakiem uzyskania satysfakcjonujących rezultatów próbowano zastosować inne wektory binarne np. otrzymane z~transformaty Fouriera sygnału oryginalnego, bądź łączących oba te wektory, lecz żadne z rozwiązań nie dawała zadowalających efektów. Zdecydowano wybrać zupełnie inny zestaw parametrów opisujących sygnał, co jest opisane w części \ref{cha:algorytm}.
