\chapter{Wykorzystany algorytm}
\label{cha:algorytm}

Zastosowany algorytm jest podobny do poprzedniego i~składa się z~następujących części: podział na fonemy, ekstrakcja cech, rozpoznanie fonemów i~dopasowanie do wzorca.

\section{Podział na fonemy}
\label{sec:podzial}

Podział na fonemy został przeprowadzony dokładnie tak samo jak w~poprzednim rozwiązaniu, z~użyciem ALCR. Należy tutaj jednak wspomnieć o~wadach tego podziału:
\begin{itemize}
    \item Nie zawsze dokładnie rozdziela fonemy w~punkcie, który podczas analizy wzrokowej uznano by za punkt graniczny.
    \item Dłuższe fonemy są często dzielone na większą liczbę fragmentów.
\end{itemize}

Wspomniano o~wadach, ponieważ mają one wpływ na jakość rozpoznawania oraz konieczność ich uzględnienia w następnych etapch algorytmu. Rośnie także złożoność obliczeniowa algorytmu.

\section{Ekstrakcja cech}
\label{sec:ekstrakcja}
Do rozpoznania wybrano cechy otrzymane z postaci czasowej i~częstotliwościowej. Wybrano 12 parametrów oraz zbadano ich zmienność zarówno w~obrębie pojedynczych fonemów, jak i~zmienność pomiędzy fonemami. Te, które określały się najlepszą zmiennością dla różnych fonemów oraz najmniejszą dla pojedynczych fonemów zostały użyte do rozpoznawania. Wszystkie zostały znormalizowane. Są to:
\begin{itemize}
    \item Liczba przejść przez zero,
    \item Stosunek największej wartości w danym fragmencie do najmniejszej,
    \item Moment spektralny zerowego rzędu,
    \item Moment znormalizowany pierwszego rzędu,
    \item Moment scentralizowany drugiego rzędu,
    \item Moment scentralizowany trzeciego rzędu,
    \item Stosunek mocy częstotliowści wysokich do częstotliwości niskich.
\end{itemize}

Wartości czasowe są liczone bez preemfazy, a~wartości częstotliwościowe są liczone z preemfazą.

\section{Rozpoznanie fonemów}
\label{sec:rozpoznanie}

Rozpoznanie fonemów odbywa się na podstawie dwóch prawdopodieństw: prawdopodobieństwa, że dany fonem wejściowy jest określonym fonemem oraz na podstawie prawdopodobieństwa, że dany fonem powinien wystąpić w danym miejscu (obliczanym na podstawie wzorców do rozpoznania).

Wyznaczono zakresy zmienności dla każdego parametru i każdego fonemu. W ten sposób otrzymano rozkład wartości parametrów dla każdego fonemu. Z rozkładu wyznaczono wartości prawdopodobieństw dla każdego parametru i fonemu. W ten sposób otrzymując na wejściu wektor cech otrzymujemy prawdopodobieństwo wystąpienia danego fonemu.

W celu wyznaczenia prawdopodobieństwa wyliczanego z wzroców oblicza się prawdopodobieństwo wystąpienia fonemu następującego po poprzednio rozpoznanym.

Do rozpoznania używana jest sieć neuronowa. Wejściami do sieci są:
\begin{itemize}
    \item Prawdopodobieństwa dla danego fonemu,
    \item Prawdopodobieństwa dla fonemu o największym prawdopodobieństwie wyliczonym za pomocą wzorców dla fonemów,
    \item Prawdopodobieństwa dla fonemu o największym prawdopodobieńśtwie wyliczonym za pomocą słów do rozpoznania,
    \item Wartości binarnej określającej czy poprzednio rozpoznany fonem jest badanym fonemem.
\end{itemize}

Ostatnia wartość polepsza działanie sieci dla długi fonemów, które składają się z kilku fragmentów.

Użyta sieć neuronowa jest siecią typu ,,Back Propagation'', z siedmiowa wejściami, trzema warstwami ukrytymi (pięć nieuronów oraz dwa neurony) oraz dwa wyjścia. Jedno wyjście określa rozpoznanie fonemu a~drugie określa brak rozpoznania fonemu. Wielkość sieci została określona na podstawie badań empirycznych.

Dla jednego fragmentu rozpoznawanego może być rozpoznanych kilka fonemów.

\section{Dopasowanie do wzorca}
\label{sec:dopasowanie}

Dopasowanie do wzorca odbywa się na podstawie sprawdzania czy kolejne rozpoznane fonemu są dopasowane do jakiegoś wzorca. Są określone następujące warunki pozwalające na dopasowanie:
\begin{itemize}
    \item Jeden fonem w wyrazie może zostać nie rozpoznany,
    \item Jednemu fonemowi we wzorcu może być dopasowanych kilka fragmentów,
    \item Mogą być fragmenty pominięte, z uwagi na szum oraz inne artefakty w sygnale.
\end{itemize}

Niestety w te sposób określone dopasowanie o wzorca ma dużą złożoność obliczeniową.
